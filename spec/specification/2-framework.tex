\chapter{Framework}\label{sec:framework}
The Radiance framework is the core component that keeps everything tied together. It should be seen as a library that provides an interaction layer for modules that are made by third-parties and don't necessarily belong to Radiance itself. In its essence it deals with providing a form to encapsulate functionality into modules and to allow for a generic interaction layer between them. \\

Interaction between modules happens through the interface and triggers system, each providing one way of interaction: Interfaces provide functions to be called on demand from a module out and triggers allow modules to hook into existing processes to add functionality or adapt data according to certain actions. \\

Being a web framework, Radiance also handles the primary dispatch and analyzing of web calls. While the web server itself is a module loaded in through the interface system, further delegation of the call happens through the framework. For this purpose, Radiance also specifies a system for dealing with requests, URLs and dispatch. \\

This section specifies the functionality grouped under the name of the radiance framework library. The specification of the necessary interfaces to complete it to a full framework follows in \nameref{sec:standard interfaces}.

\section{Modules}\label{sec:modules}
To offer a way to allow extension of the framework itself as well as web applications in a unified way, Radiance uses a system of modules that encapsulates functionality. Each module is defined in the context of a package, an ASDF system and a unique identifier. Each of these components is linked to each other and should allow for identification, loading and extension of a module. In particular the unique identifier is of importance to the interface system and its dispatch mechanism.
\subsection{Overview}
\subsection{ASDF}\label{sec:mod asdf}
\subsection{Module Identifiers}\label{sec:mod module identifiers}

\section{Interfaces}\label{sec:int interfaces}
\subsection{Definition}\label{sec:int definition}
\subsection{Interface Extensions}\label{sec:int interface extensions}
\subsection{Component Expanders}\label{sec:int component expanders}
\subsection{Method Definition}\label{sec:int method definition}
\subsection{ASDF Loading}\label{sec:int asdf loading}

\section{Triggers}\label{sec:triggers}
\subsection{Namespaces}\label{sec:trig namespaces}
\subsection{Hooks}\label{sec:trig hooks}
\subsection{Triggers}\label{sec:trig triggers}

\section{URI}\label{sec:uri}
\subsection{Properties}\label{sec:uri properties}
\subsection{Matching}\label{sec:uri matching}

\section{Request Continuations}\label{sec:request continuations}
\subsection{Purpose}\label{sec:req purpose}
\subsection{Invocation}\label{sec:req invocation}
\subsection{Extent of Use}\label{sec:req extent of use}

\section{Server}\label{sec:server}
\subsection{Interface}\label{sec:ser interface}
\subsection{Management}\label{sec:ser management}
\subsection{Request Handling}\label{sec:ser request handling}

%%% Local Variables: 
%%% mode: latex
%%% TeX-master: "master"
%%% End: 
